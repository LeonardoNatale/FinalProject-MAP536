\documentclass[a4paper,12pt,twoside]{article}
\usepackage[T1]{fontenc}
\usepackage[utf8]{inputenc}
\usepackage{lmodern}
\usepackage{url,csquotes}
\usepackage[hidelinks,hyperfootnotes=false]{hyperref}
\usepackage[titlepage,fancysections,pagenumber]{polytechnique}


\title{MAP-536}
\subtitle{Final Project}
\author{Leonardo \textsc{Natale} \& Guillaume \textsc{Le Fur}}
\logo{logo.pdf}

\begin{document}

\maketitle

\section{The Task}
Approach:
\begin{itemize}
	\item Explore different ML models
	\item Data preprocessing, and external datasets
	\item Tune model hyper parameters \\

\end{itemize}
Bonus (advanced):
\begin{itemize}
	\item Model interpretability
	\item Evaluate uncertainty in predictions
\end{itemize}

\section{Data Manipulation}

\subsection{Getting Extra Data}
The original data consists of:
\begin{itemize}
	\item the date of departure
	\item the departure and arrival airport
	\item the mean and standard deviation of the number of weeks of the reservations made before the departure date
    \item a field called log\_PAX which is related to the number of passengers (the actual number were changed for privacy reasons)
\end{itemize}

\subsection{Feature Engineering}

\section{Installation du package}

\subsection{Prérequis}

\subsubsection{Packages requis}



\subsubsection{Installer les packages manquants}


\paragraph{Mac} Pour mettre à jour ou installer des packages, il faut utiliser l'application \emph{Tex Live utility}, dont un tutoriel en français très 
\paragraph{Linux} Dans un terminal, exécuter la commande 
\begin{verbatim}
tlmgr install <nom du package>
\end{verbatim}


\subsection{Installation du package \texttt{polytechnique}}

\subsubsection{Méthode automatique}



Extraire le dossier \texttt{polytechnique-LaTeX} à la racine de l'archive quelque part sur son ordinateur. Entrer dans le dossier et effectuer l'action suivante :
\begin{description}
\item[Windows] double-cliquer sur \texttt{makefile\_windows.bat} ;
\item[Linux ou Mac] ouvrir un terminal dans le dossier et y entrer la commande
\begin{verbatim}
make && make install
\end{verbatim}
\end{description}
Après exécution du script, le message affiché sur la console devrait être :
\begin{verbatim}
************************************************
* Le package a ete installe ! La documentation *
* se trouve a l'emplacement suivant :          *
<chemin qui dépend de votre ordinateur>
* Les fichiers sources et resources a          *
* l'endroit suivant :                          *
<chemin qui dépend de votre ordinateur>
* Et le .sty a l'endroit suivant :             *
<chemin qui dépend de votre ordinateur>
* Bonne utilisation !                          *
************************************************
\end{verbatim}

Si ce message ne s'affiche pas ou si l'installation semble ne pas fonctionner quand vous essayez d'utiliser le package, se reporter à la méthode manuelle.
\subsubsection{Méthode manuelle}


\subsection{Documentation}

La documentation du package est le fichier \texttt{polytechnique.pdf} qui se trouve entre autres dans le dossier \emph{source} de l'archive extraite.

\end{document}
